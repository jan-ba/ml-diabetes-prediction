\documentclass[a4paper,12pt]{article}
\usepackage[utf8]{inputenc}
\usepackage[english]{babel}
\usepackage{graphicx}
\usepackage{amsmath}
\usepackage{amssymb}
\usepackage{hyperref}
\usepackage{geometry}
\usepackage{tabularx}
\usepackage[backend=biber, style=numeric, sorting=none]{biblatex}
\addbibresource{references.bib}

\geometry{a4paper, left=25mm, right=25mm, top=25mm, bottom=25mm}
\title{Project 1: Supervised Learning - Classification}
\author{Alexander Svarfdal Gudmundsson \and Jan Babin}
\date{\today}

\begin{document}

\maketitle

\begin{abstract}
Placeholder. We probably dont need an abstract since this is a report rather than a paper.
// REMINDER TO JOIN A SUBMISSION GROUP ON CANVAS
\end{abstract}

\section{Introduction}
\subsection{Background}
Diabetes is a chronic condition characterised by elevated blood sugar levels. While genetic factors 
can play a role, a significant amount of diabetes Type 2 cases is associated with lifestyle choices. 
According to the World Health Organization \cite{WHO2016}, 422 million people worldwide 
are living with diabetes, with numbers continuing to rise. The societal impact of diabetes is 
significant: a diminished quality of life and a higher risk of serious health complications which in 
turn may lead to increased healthcare costs. Therefore, there is medical merit in
early predictions of diabetes to refine treatment and management strategies.

\subsection{Objective}
The goal of this project is to build a machine learning model that can identify 
patterns in lifestyle and demographic data associated with diabetes and thereby predict whether
a given person has diabetes / is very likely do develop it. While the 
model is purely intended for academic purpose, the project allows us to 
explore how machine learning can be used to analyse health-related data and 
provide insights into potential risk factors.

\subsection{Dataset}
The dataset at hand, "Diabetes, Hypertension and Stroke Prediction," is from Kaggle and contains 
health-related data in CSV format intended for predicting diabetes, hypertension and stroke. 
For this project, we are solely focussing on diabetes. The dataset consists of 70,692 observations 
and 18 attributes, some of which are outlined as follows (full overview \ref{tab:feature_list}):
\begin{itemize}
    \item \textbf{Age}: Coded in 13 age groups (e.g., 1.0 for 18-24, 2.0 for 25-29, etc.).
    \item \textbf{Sex}: Binary variable representing male (0.0) and female (1.0).
    \item \textbf{BMI}: Body Mass Index, a continuous variable.
    \item \textbf{Lifestyle indicators}: Such as smoking status (whether the individual has smoked at least 100 cigarettes in their lifetime), physical activity in the past 30 days (excluding job-related activity), daily fruit and vegetable consumption, and heavy alcohol consumption (based on defined weekly limits for men and women).
    \item \textbf{General health and mental/physical health}: Self-reported general health on a scale from 1 (excellent) to 5 (poor), along with the number of days with poor mental and physical health over the past 30 days.
    \item \textbf{Pre-existing conditions}: Information on high cholesterol, coronary heart disease or myocardial infarction, and difficulty walking.
\end{itemize}

The target variable is \textbf{Diabetes}, which is represented as a binary category 
(0.0 for no diabetes, 1.0 for diabetes). No missing or null values were identified in the dataset, 
ensuring that data cleaning was minimal and straightforward. Furthermore, the dataset is perfectly 
balanced, in that, 50\% of observations correspond to diabetes, the other 50\% do not.

\section{Process}
The steps involved in the analysis follow a standard supervised machine learning pipeline, 
from data preparation to model selection and evaluation.

\subsection{Data Loading and Exploration}
The dataset was loaded into a pandas DataFrame. Initial exploration was conducted to understand 
the structure of the data:
\begin{itemize}
    \item Data Integrity: We confirmed that the dataset contained no missing or null values.
    \item Feature Exploration: The features were analysed for their unique values and data 
    types, revealing that many variables were binary, and the rest were either categorical or 
    continuous. Continuous variables included features like BMI, Age, MentHlth, and PhysHlth, 
    while categorical variables included GenHlth (self-reported health scale from 1 to 5) 
    and binary indicators for lifestyle factors like Smoking, Physical Activity, and Alcohol 
    Consumption.
\end{itemize}

\subsection{Preprocessing}
\subsubsection{One-Hot Encoding}
WE SHOULD DISCUSS THIS AGAIN

Since some categorical features (e.g., Sex, GenHlth) were non-numeric, one-hot encoding was 
applied to convert these categories into numeric values:
\begin{itemize}
    \item Sex: Encoded as a binary variable with one-hot encoding and drop\_first=True to prevent
    perfect multicollinearity (i.e., avoiding redundancy with one binary variable sufficing to 
    represent gender).
    \item GenHlth: This feature, originally a categorical variable on a scale from 1 to 5, 
    was one-hot encoded to allow for better model interpretation and training.
\end{itemize}

\subsubsection{Scaling}
Certain features with a wide range of values were standardized using StandardScaler. These included:
BMI, MentHlth, PhysHlth, and Age: Standardization ensures that the models 
(e.g., logistic regression, support vector machines) are not biased towards features with larger 
numeric ranges. This is important for algorithms that rely on the magnitude of features during 
decision-making.

\subsubsection{Train-Test Split}
WE MIGHT WANNA EXPLAIN HERE WHY WE CHOSE THIS SPLIT

The dataset was split into a training set (80\%) and a test set (20\%) using train\_test\_split from 
sklearn. Stratification was applied to ensure that the class distribution of the target variable 
(Diabetes) remained balanced across the training and test sets.

\subsection{Feature Correlation Analysis}
To explore potential multicollinearity among features, a correlation matrix was generated using 
Seaborn's heatmap function. The goal was to identify highly correlated features that could negatively 
impact model performance by redundancy.

Findings: The GenHlth and PhysHlth features exhibited some degree of linear correlation but it was
not high enough to justify omitting one of two would outweigh having more features / data to train 
on.
WE MIGHT WANNA REFERENC SOMETHING REGARDING CORRELATION VALUES HERE 
(when to throw away a feature / when not)

\subsection{Model Selection and Evaluation}
\subsubsection{PyCaret Experiment Setup}
For model selection, we utilised PyCaret's ClassificationExperiment to quickly get a 
comparison between multiple classification models. We then selected the 
5 best performing models based on default hyperparameters to fine-tune further via hyperparameter
optimisation.
\\
PyCaret’s built-in random grid-search was used to tune the hyperparameters of the selected models, 
optimising for XXXXXXXX.

\subsubsection{Model Stacking and Ensembling}
Ensemble Methods: In an effort to improve prediction accuracy, ensemble learning methods were applied. 
A hard voting ensemble (WHAT DOES THIS EXACTLY DO?) (blend\_models) was created using the 
top-performing models.
Stacking: A stacking model was also built, ..... 
However, the stacking model did not outperform the ensemble.

\subsection{Model Stacking and Ensembling}
After experimenting with feature selection (also implemented via PyCaret), we found that applying 
feature selection resulted in slightly lower model performance. Therefore, the final models were 
built without this step.

\section{Results}
Placeholder

\section{Conclusions}
Placeholder

\section{Future Work}
Placeholder

\clearpage

\appendix
\section*{Appendix}

\begin{table}[h!]
    \centering
    \begin{tabularx}{\textwidth}{|l|X|}
    \hline
    \textbf{Feature} & \textbf{Description} \\ \hline
    Age & Coded in 13 age groups (e.g., 1: 18-24, 2: 25-29, etc.) \\ \hline
    Sex & Sex of the individual (0: Male, 1: Female) \\ \hline
    HighChol & High cholesterol (0: No, 1: Yes) \\ \hline
    CholCheck & Checked cholesterol in the last 5 years (0: No, 1: Yes) \\ \hline
    BMI & Body Mass Index (continuous variable) \\ \hline
    Smoker & Smoked at least 100 cigarettes in their lifetime (0: No, 1: Yes) \\ \hline
    HeartDiseaseorAttack & History of coronary heart disease or myocardial infarction (0: No, 1: Yes) \\ \hline
    PhysActivity & Engaged in physical activity in the past 30 days, excluding work (0: No, 1: Yes) \\ \hline
    Fruits & Consumes fruit 1 or more times per day (0: No, 1: Yes) \\ \hline
    Veggies & Consumes vegetables 1 or more times per day (0: No, 1: Yes) \\ \hline
    HvyAlcoholConsump & Heavy alcohol consumption (men: 14+ drinks/week, women: 7+ drinks/week) (0: No, 1: Yes) \\ \hline
    GenHlth & Self-reported general health (1: Excellent, 2: Very good, 3: Good, 4: Fair, 5: Poor) \\ \hline
    MentHlth & Days of poor mental health in the past 30 days (0 to 30) \\ \hline
    PhysHlth & Days of poor physical health in the past 30 days (0 to 30) \\ \hline
    DiffWalk & Difficulty walking or climbing stairs (0: No, 1: Yes) \\ \hline
    Stroke & History of stroke (0: No, 1: Yes) \\ \hline
    HighBP & High blood pressure (0: No, 1: Yes) \\ \hline
    Diabetes & Presence of diabetes (0: No, 1: Yes) \\ \hline
    \end{tabularx}
    \caption{Full list of dataset features used in the analysis.}
    \label{tab:feature_list}
\end{table}

\printbibliography

\end{document}